\section{Research Method}
	This section describes the research method that was used.

	\subsection{Literature Study}
		In this section a description of the process by which literature was collected for the purpose of this study is provided to ensure its replicability.

		\subsubsection{Search}
			Google Scholar\footnote{\url{www.scholar.google.com}} was used as the main tool for discovering relevant literature.
			Google Scholar works similarly to Google in that it uses a search query and presents relevant results depending on the input.
		
			Some keywords were extracted from the research questions to be used as the base for the search query:
			\begin{itemize}
				\item \acrshort{cpu}
				\item \acrshort{gpu}
				\item workload/usage pattern
				\item analysis/analyzing/characterize/characterization/model/modeling
				\item energy/power
				\item saving/conservation
			\end{itemize}
		
			From these keywords, and the support Google Scholar has for advanced search operators such as the Boolean OR operator to provide constraints to the search query \parencite{Russel}, the following search queries were then constructed:
			\begin{aligneddescription}{SQ1}
				\item[SQ1] \sqOne
				\item[SQ2] \sqTwo
			\end{aligneddescription}
			
		\subsubsection{Gathering Literature}
			To collect literature a software tool, Mendeley \footnote{\url{www.mendeley.com}}, was used.
			To organize the literature I subdivided the literature into four categories, as follows:
			\begin{aligneddescription}{Not selected}
				\item[Unread]       Literature that was gathered from a search query but that has not yet been read.
				\item[Related]      Literature that has been read and is indirectly related to this literature review.
				\item[Selected]     Literature that was read and that matches the inclusion criteria.
				\item[Not selected] Literature that was read but that does not match the inclusion criteria.
			\end{aligneddescription}

			\paragraph{Snowballing}
				% If a snowballing approach was used for the literature review, its details should be documented in this subsection
				To gather more relevant literature the snowballing technique was used, which is the process of gathering additional literature from the references of a paper.

		\subsubsection{Application of Selection Criteria}
			% In this section, the selection criteria, in terms of inclusion and inclusion criteria should be documented.
			% See examples below, which should be adapted to your specific case and, in any case, *rephrased in our own words*
			I used a set of selection criteria to filter out any irrelevant papers from the search.
			These criteria can be subdivided into inclusion and exclusion criteria.
			This section outlines those criteria and provides the reasoning behind them.
			
			\paragraph{Inclusion Criteria}
				At least one of these inclusion criteria must be fulfilled by each of the papers selected:
			
				\begin{aligneddescription}{IC1}
					\item[IC1] \icOne
					\item[IC2] \icTwo
					\item[IC3] \icThree
				\end{aligneddescription}
				
			\paragraph{Exclusion Criteria}
				None of these exclusion criteria must be fulfilled by each of the papers selected:
			
				\begin{aligneddescription}{EC1}
					\item[EC1] \ecOne
				\end{aligneddescription}
		
		\subsubsection{Data Extraction}
			% Report in this section the data extraction followed to gather the data for the study (e.g., what process did you followed to gather the data in the companion data extraction spreadsheet?)
			TODO
		
		\subsubsection{Data Synthesis}
			% What approach did you follow to carry out the data synthesis process and summarize the data extracted from the primary studies?
			TODO
