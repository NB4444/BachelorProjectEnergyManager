\section{Research Method}
	This section describes the research method that was used.

	\subsection{Literature Study}
		In this section we first provide a description of the process by which literature was collected for the purpose of this study to ensure its replicability.

		\subsubsection{Search}
			The search for literature was conducted using Google Scholar\footnote{\url{www.scholar.google.com}}.
			Google Scholar works similarly to Google in that it uses a search query and presents relevant results depending on the input.
			Search queries on Google Scholar can be built from keywords and Boolean operators such as OR to provide constraints to the search query \parencite{Russel}.
		
			To build the search query, keywords are extracted from the research questions so that those keywords can be used as a base for a search query on Google Scholar.
			This results in the following keywords along with any potential synonyms:
			\begin{itemize}
				\item \acrshort{cpu}
				\item \acrshort{gpu}
				\item workload
				\item analysis/analyzing/characterize/characterization
				\item energy/power
				\item saving/conservation
			\end{itemize}
		
			From these keywords the following search queries were then constructed:
			\begin{aligneddescription}{SQ1}
				\item[SQ1] \sqOne
				\item[SQ2] \sqTwo
			\end{aligneddescription}
			
		\subsubsection{Gathering Literature}
			To initial round of literature gathering was performed with the assistance of a software tool used for bibliography management called Mendeley \footnote{\url{www.mendeley.com}}.
			In this tool, four categories were created to organize the literature:
			\begin{aligneddescription}{Not selected}
				\item[Unread]       Literature that was gathered from a search query but that has not yet been read.
				\item[Related]      Literature that has been read and is indirectly related to this literature review.
				\item[Selected]     Literature that was read and that matches the inclusion criteria.
				\item[Not selected] Literature that was read but that does not match the inclusion criteria.
			\end{aligneddescription}
			All papers that were found during the initial search were placed in the unread category, after which they were moved to another category depending on the contents of the paper and their applicability to the topic of this literature review.

			\paragraph{Snowballing}
				% If a snowballing approach was used for the literature review, its details should be documented in this subsection
				To gather more relevant literature the snowballing technique was used, which is the process of gathering additional literature from the references of a paper.

		\subsubsection{Application of Selection Criteria}
			% In this section, the selection criteria, in terms of inclusion and inclusion criteria should be documented.
			% See examples below, which should be adapted to your specific case and, in any case, *rephrased in our own words*
			In order to restrict the amount of papers that need to be processed and to filter out any irrelevant papers selection criteria were used.
			To this end, the literature review process was conducted by looking for papers that fulfil all of the specified inclusion criteria while matching none of the specified exclusion criteria.
			This section outlines those criteria and the reasoning behind them.
			Most of these criteria were sourced from the research questions and are meant to help answer them.
			
			\paragraph{Inclusion Criteria}
				At least one of these inclusion criteria must be fulfilled by each of the papers selected:
			
				\begin{aligneddescription}{IC1}
					\item[IC1] \icOne
					\item[IC2] \icTwo
					\item[IC3] \icThree
				\end{aligneddescription}
				
			\paragraph{Exclusion Criteria}
				None of these exclusion criteria must be fulfilled by each of the papers selected:
			
				\begin{aligneddescription}{EC1}
					\item[EC1] \ecOne
				\end{aligneddescription}
		
		\subsubsection{Data Extraction}
			% Report in this section the data extraction followed to gather the data for the study (e.g., what process did you followed to gather the data in the companion data extraction spreadsheet?)
			TODO
		
		\subsubsection{Data Synthesis}
			% What approach did you follow to carry out the data synthesis process and summarize the data extracted from the primary studies?
			TODO
