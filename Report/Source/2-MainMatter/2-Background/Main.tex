\chapter{Background}
	This chapter outlines some of the research and other resources that are relevant to the topic of \gls{gpu} energy conservation.

	\section{Energy Consumption}
		This section outlines some of the work that has been done to measure and predict energy consumption.

		\subsection{Measuring}
			Measuring live energy consumption is an important aspect of many power saving strategies.
			There exist tools that can perform these types of measurement, the most important of which are outlined in this section.

			\subsubsection{NVIDIA \acrlong{smi}}
				NVIDIA's \gls{smi} tool is a command line utility that is able to query the \gls{gpu} device state \parencite{NVIDIA}.
				Support is limited to NVIDIA \glspl{gpu}.
				What makes this tool useful to this research is the fact that it can retrieve the current power consumption from the \gls{gpu} as it is running and that it can output this information to the console, which makes it possible to easily integrate the output programmatically.

		\subsection{Workload Analysis}
			An important component in any energy saving strategy is to perform a workload analysis, since the decisions that are made often depend on the type of workload that is running \parencite{Chen2011}.

			\subsubsection{GPGPUSim}
				\emph{GPGPUSim} is a tool that can be used to simulate a \gls{gpu} and run synthetic workloads.
				It offers a lot of detailed insights that can be used for workload analysis \parencite{Bakhoda2009}.

			\subsubsection{Usage Patterns}
				% List some common usage patterns found in the workload analysis here
				TODO

		\subsection{Statistical Analysis and Prediction}
			\textcite{Ma2009} developed a method to statistically analyze and model the power consumption of a mainstream \gls{gpu}.
			To achieve this they make use of the fact that there exists an innate coupling among the power consumption characteristics, runtime performance and dynamic workloads.
			They found that their model is capable of robustly and accurately predicting the dynamic power consumption estimation of a target \gls{gpu} at runtime, especially for graphics applications.
			
			\textcite{Ma2009} state that due to the relatively simpler cache hierarchy, higher level of parallelism, less complex control requirements, and more computation units, \gls{gpu} power modeling differs from general-purpose processing units.
			Some limitations of their approach they state are that micro architectural knowledge of the \gls{gpu} is needed to provide more complex and accurate modeling approaches, and that quantitative analysis of \gls{gpu} workloads and statistical selection of the power consumption correlated workloads are necessary in the data preprocessing step.

			\textcite{Chen2011} also developed a method to statistically analyze \gls{gpu} power consumption.
			They designed a high-level \gls{gpu} power consumption model using sophisticated tree-based random forest methods which can correlate the power consumption with a set of independent performance variables.
			Their model is able to accurately predict \gls{gpu} runtime power consumption and provides insights for understanding the dependence between the \gls{gpu} runtime power consumption and the individual performance metrics.
			To gain detailed insights they used a \gls{gpu} simulator, \emph{GPGPUSim} \parencite{Bakhoda2009}.

	\section{Energy Saving}
		TODO

		\subsection{\acrlong{dvfs}}
			\gls{dvfs} is a technique that